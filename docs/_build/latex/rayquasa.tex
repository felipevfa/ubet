% Generated by Sphinx.
\def\sphinxdocclass{report}
\documentclass[letterpaper,10pt,english]{sphinxmanual}

\usepackage[utf8]{inputenc}
\ifdefined\DeclareUnicodeCharacter
  \DeclareUnicodeCharacter{00A0}{\nobreakspace}
\else\fi
\usepackage{cmap}
\usepackage[T1]{fontenc}
\usepackage{amsmath,amssymb}
\usepackage{babel}
\usepackage{times}
\usepackage[Bjarne]{fncychap}
\usepackage{longtable}
\usepackage{sphinx}
\usepackage{multirow}
\usepackage{eqparbox}


\addto\captionsenglish{\renewcommand{\figurename}{Fig. }}
\addto\captionsenglish{\renewcommand{\tablename}{Table }}
\SetupFloatingEnvironment{literal-block}{name=Listing }

\addto\extrasenglish{\def\pageautorefname{page}}

\setcounter{tocdepth}{1}


\title{rayquasa Documentation}
\date{May 12, 2016}
\release{0.1}
\author{nilo}
\newcommand{\sphinxlogo}{}
\renewcommand{\releasename}{Release}
\makeindex

\makeatletter
\def\PYG@reset{\let\PYG@it=\relax \let\PYG@bf=\relax%
    \let\PYG@ul=\relax \let\PYG@tc=\relax%
    \let\PYG@bc=\relax \let\PYG@ff=\relax}
\def\PYG@tok#1{\csname PYG@tok@#1\endcsname}
\def\PYG@toks#1+{\ifx\relax#1\empty\else%
    \PYG@tok{#1}\expandafter\PYG@toks\fi}
\def\PYG@do#1{\PYG@bc{\PYG@tc{\PYG@ul{%
    \PYG@it{\PYG@bf{\PYG@ff{#1}}}}}}}
\def\PYG#1#2{\PYG@reset\PYG@toks#1+\relax+\PYG@do{#2}}

\expandafter\def\csname PYG@tok@gd\endcsname{\def\PYG@tc##1{\textcolor[rgb]{0.63,0.00,0.00}{##1}}}
\expandafter\def\csname PYG@tok@gu\endcsname{\let\PYG@bf=\textbf\def\PYG@tc##1{\textcolor[rgb]{0.50,0.00,0.50}{##1}}}
\expandafter\def\csname PYG@tok@gt\endcsname{\def\PYG@tc##1{\textcolor[rgb]{0.00,0.27,0.87}{##1}}}
\expandafter\def\csname PYG@tok@gs\endcsname{\let\PYG@bf=\textbf}
\expandafter\def\csname PYG@tok@gr\endcsname{\def\PYG@tc##1{\textcolor[rgb]{1.00,0.00,0.00}{##1}}}
\expandafter\def\csname PYG@tok@cm\endcsname{\let\PYG@it=\textit\def\PYG@tc##1{\textcolor[rgb]{0.25,0.50,0.56}{##1}}}
\expandafter\def\csname PYG@tok@vg\endcsname{\def\PYG@tc##1{\textcolor[rgb]{0.73,0.38,0.84}{##1}}}
\expandafter\def\csname PYG@tok@vi\endcsname{\def\PYG@tc##1{\textcolor[rgb]{0.73,0.38,0.84}{##1}}}
\expandafter\def\csname PYG@tok@mh\endcsname{\def\PYG@tc##1{\textcolor[rgb]{0.13,0.50,0.31}{##1}}}
\expandafter\def\csname PYG@tok@cs\endcsname{\def\PYG@tc##1{\textcolor[rgb]{0.25,0.50,0.56}{##1}}\def\PYG@bc##1{\setlength{\fboxsep}{0pt}\colorbox[rgb]{1.00,0.94,0.94}{\strut ##1}}}
\expandafter\def\csname PYG@tok@ge\endcsname{\let\PYG@it=\textit}
\expandafter\def\csname PYG@tok@vc\endcsname{\def\PYG@tc##1{\textcolor[rgb]{0.73,0.38,0.84}{##1}}}
\expandafter\def\csname PYG@tok@il\endcsname{\def\PYG@tc##1{\textcolor[rgb]{0.13,0.50,0.31}{##1}}}
\expandafter\def\csname PYG@tok@go\endcsname{\def\PYG@tc##1{\textcolor[rgb]{0.20,0.20,0.20}{##1}}}
\expandafter\def\csname PYG@tok@cp\endcsname{\def\PYG@tc##1{\textcolor[rgb]{0.00,0.44,0.13}{##1}}}
\expandafter\def\csname PYG@tok@gi\endcsname{\def\PYG@tc##1{\textcolor[rgb]{0.00,0.63,0.00}{##1}}}
\expandafter\def\csname PYG@tok@gh\endcsname{\let\PYG@bf=\textbf\def\PYG@tc##1{\textcolor[rgb]{0.00,0.00,0.50}{##1}}}
\expandafter\def\csname PYG@tok@ni\endcsname{\let\PYG@bf=\textbf\def\PYG@tc##1{\textcolor[rgb]{0.84,0.33,0.22}{##1}}}
\expandafter\def\csname PYG@tok@nl\endcsname{\let\PYG@bf=\textbf\def\PYG@tc##1{\textcolor[rgb]{0.00,0.13,0.44}{##1}}}
\expandafter\def\csname PYG@tok@nn\endcsname{\let\PYG@bf=\textbf\def\PYG@tc##1{\textcolor[rgb]{0.05,0.52,0.71}{##1}}}
\expandafter\def\csname PYG@tok@no\endcsname{\def\PYG@tc##1{\textcolor[rgb]{0.38,0.68,0.84}{##1}}}
\expandafter\def\csname PYG@tok@na\endcsname{\def\PYG@tc##1{\textcolor[rgb]{0.25,0.44,0.63}{##1}}}
\expandafter\def\csname PYG@tok@nb\endcsname{\def\PYG@tc##1{\textcolor[rgb]{0.00,0.44,0.13}{##1}}}
\expandafter\def\csname PYG@tok@nc\endcsname{\let\PYG@bf=\textbf\def\PYG@tc##1{\textcolor[rgb]{0.05,0.52,0.71}{##1}}}
\expandafter\def\csname PYG@tok@nd\endcsname{\let\PYG@bf=\textbf\def\PYG@tc##1{\textcolor[rgb]{0.33,0.33,0.33}{##1}}}
\expandafter\def\csname PYG@tok@ne\endcsname{\def\PYG@tc##1{\textcolor[rgb]{0.00,0.44,0.13}{##1}}}
\expandafter\def\csname PYG@tok@nf\endcsname{\def\PYG@tc##1{\textcolor[rgb]{0.02,0.16,0.49}{##1}}}
\expandafter\def\csname PYG@tok@si\endcsname{\let\PYG@it=\textit\def\PYG@tc##1{\textcolor[rgb]{0.44,0.63,0.82}{##1}}}
\expandafter\def\csname PYG@tok@s2\endcsname{\def\PYG@tc##1{\textcolor[rgb]{0.25,0.44,0.63}{##1}}}
\expandafter\def\csname PYG@tok@nt\endcsname{\let\PYG@bf=\textbf\def\PYG@tc##1{\textcolor[rgb]{0.02,0.16,0.45}{##1}}}
\expandafter\def\csname PYG@tok@nv\endcsname{\def\PYG@tc##1{\textcolor[rgb]{0.73,0.38,0.84}{##1}}}
\expandafter\def\csname PYG@tok@s1\endcsname{\def\PYG@tc##1{\textcolor[rgb]{0.25,0.44,0.63}{##1}}}
\expandafter\def\csname PYG@tok@ch\endcsname{\let\PYG@it=\textit\def\PYG@tc##1{\textcolor[rgb]{0.25,0.50,0.56}{##1}}}
\expandafter\def\csname PYG@tok@m\endcsname{\def\PYG@tc##1{\textcolor[rgb]{0.13,0.50,0.31}{##1}}}
\expandafter\def\csname PYG@tok@gp\endcsname{\let\PYG@bf=\textbf\def\PYG@tc##1{\textcolor[rgb]{0.78,0.36,0.04}{##1}}}
\expandafter\def\csname PYG@tok@sh\endcsname{\def\PYG@tc##1{\textcolor[rgb]{0.25,0.44,0.63}{##1}}}
\expandafter\def\csname PYG@tok@ow\endcsname{\let\PYG@bf=\textbf\def\PYG@tc##1{\textcolor[rgb]{0.00,0.44,0.13}{##1}}}
\expandafter\def\csname PYG@tok@sx\endcsname{\def\PYG@tc##1{\textcolor[rgb]{0.78,0.36,0.04}{##1}}}
\expandafter\def\csname PYG@tok@bp\endcsname{\def\PYG@tc##1{\textcolor[rgb]{0.00,0.44,0.13}{##1}}}
\expandafter\def\csname PYG@tok@c1\endcsname{\let\PYG@it=\textit\def\PYG@tc##1{\textcolor[rgb]{0.25,0.50,0.56}{##1}}}
\expandafter\def\csname PYG@tok@o\endcsname{\def\PYG@tc##1{\textcolor[rgb]{0.40,0.40,0.40}{##1}}}
\expandafter\def\csname PYG@tok@kc\endcsname{\let\PYG@bf=\textbf\def\PYG@tc##1{\textcolor[rgb]{0.00,0.44,0.13}{##1}}}
\expandafter\def\csname PYG@tok@c\endcsname{\let\PYG@it=\textit\def\PYG@tc##1{\textcolor[rgb]{0.25,0.50,0.56}{##1}}}
\expandafter\def\csname PYG@tok@mf\endcsname{\def\PYG@tc##1{\textcolor[rgb]{0.13,0.50,0.31}{##1}}}
\expandafter\def\csname PYG@tok@err\endcsname{\def\PYG@bc##1{\setlength{\fboxsep}{0pt}\fcolorbox[rgb]{1.00,0.00,0.00}{1,1,1}{\strut ##1}}}
\expandafter\def\csname PYG@tok@mb\endcsname{\def\PYG@tc##1{\textcolor[rgb]{0.13,0.50,0.31}{##1}}}
\expandafter\def\csname PYG@tok@ss\endcsname{\def\PYG@tc##1{\textcolor[rgb]{0.32,0.47,0.09}{##1}}}
\expandafter\def\csname PYG@tok@sr\endcsname{\def\PYG@tc##1{\textcolor[rgb]{0.14,0.33,0.53}{##1}}}
\expandafter\def\csname PYG@tok@mo\endcsname{\def\PYG@tc##1{\textcolor[rgb]{0.13,0.50,0.31}{##1}}}
\expandafter\def\csname PYG@tok@kd\endcsname{\let\PYG@bf=\textbf\def\PYG@tc##1{\textcolor[rgb]{0.00,0.44,0.13}{##1}}}
\expandafter\def\csname PYG@tok@mi\endcsname{\def\PYG@tc##1{\textcolor[rgb]{0.13,0.50,0.31}{##1}}}
\expandafter\def\csname PYG@tok@kn\endcsname{\let\PYG@bf=\textbf\def\PYG@tc##1{\textcolor[rgb]{0.00,0.44,0.13}{##1}}}
\expandafter\def\csname PYG@tok@cpf\endcsname{\let\PYG@it=\textit\def\PYG@tc##1{\textcolor[rgb]{0.25,0.50,0.56}{##1}}}
\expandafter\def\csname PYG@tok@kr\endcsname{\let\PYG@bf=\textbf\def\PYG@tc##1{\textcolor[rgb]{0.00,0.44,0.13}{##1}}}
\expandafter\def\csname PYG@tok@s\endcsname{\def\PYG@tc##1{\textcolor[rgb]{0.25,0.44,0.63}{##1}}}
\expandafter\def\csname PYG@tok@kp\endcsname{\def\PYG@tc##1{\textcolor[rgb]{0.00,0.44,0.13}{##1}}}
\expandafter\def\csname PYG@tok@w\endcsname{\def\PYG@tc##1{\textcolor[rgb]{0.73,0.73,0.73}{##1}}}
\expandafter\def\csname PYG@tok@kt\endcsname{\def\PYG@tc##1{\textcolor[rgb]{0.56,0.13,0.00}{##1}}}
\expandafter\def\csname PYG@tok@sc\endcsname{\def\PYG@tc##1{\textcolor[rgb]{0.25,0.44,0.63}{##1}}}
\expandafter\def\csname PYG@tok@sb\endcsname{\def\PYG@tc##1{\textcolor[rgb]{0.25,0.44,0.63}{##1}}}
\expandafter\def\csname PYG@tok@k\endcsname{\let\PYG@bf=\textbf\def\PYG@tc##1{\textcolor[rgb]{0.00,0.44,0.13}{##1}}}
\expandafter\def\csname PYG@tok@se\endcsname{\let\PYG@bf=\textbf\def\PYG@tc##1{\textcolor[rgb]{0.25,0.44,0.63}{##1}}}
\expandafter\def\csname PYG@tok@sd\endcsname{\let\PYG@it=\textit\def\PYG@tc##1{\textcolor[rgb]{0.25,0.44,0.63}{##1}}}

\def\PYGZbs{\char`\\}
\def\PYGZus{\char`\_}
\def\PYGZob{\char`\{}
\def\PYGZcb{\char`\}}
\def\PYGZca{\char`\^}
\def\PYGZam{\char`\&}
\def\PYGZlt{\char`\<}
\def\PYGZgt{\char`\>}
\def\PYGZsh{\char`\#}
\def\PYGZpc{\char`\%}
\def\PYGZdl{\char`\$}
\def\PYGZhy{\char`\-}
\def\PYGZsq{\char`\'}
\def\PYGZdq{\char`\"}
\def\PYGZti{\char`\~}
% for compatibility with earlier versions
\def\PYGZat{@}
\def\PYGZlb{[}
\def\PYGZrb{]}
\makeatother

\renewcommand\PYGZsq{\textquotesingle}

\begin{document}

\maketitle
\tableofcontents
\phantomsection\label{index::doc}


a
Contents:


\chapter{Models}
\label{modules/models:models}\label{modules/models::doc}\label{modules/models:welcome-to-rayquasa-s-documentation}\label{modules/models:module-ubet.models}\index{ubet.models (module)}\index{Admin\_settings (class in ubet.models)}

\begin{fulllineitems}
\phantomsection\label{modules/models:ubet.models.Admin_settings}\pysiglinewithargsret{\strong{class }\code{ubet.models.}\bfcode{Admin\_settings}}{\emph{*args}, \emph{**kwargs}}{}
Configuracoes gerais do servico. Para melhor desempenho, precisam ser armazenadas em cache
\index{time\_to\_expire (ubet.models.Admin\_settings attribute)}

\begin{fulllineitems}
\phantomsection\label{modules/models:ubet.models.Admin_settings.time_to_expire}\pysigline{\bfcode{time\_to\_expire}\strong{ = None}}
tempo de duracao maxima de um grupo

\end{fulllineitems}

\index{win\_tax (ubet.models.Admin\_settings attribute)}

\begin{fulllineitems}
\phantomsection\label{modules/models:ubet.models.Admin_settings.win_tax}\pysigline{\bfcode{win\_tax}\strong{ = None}}
um valor entre 0 e 1 que representa o percentual de comissao sobre o premio
do vencedor

\end{fulllineitems}


\end{fulllineitems}

\index{Group (class in ubet.models)}

\begin{fulllineitems}
\phantomsection\label{modules/models:ubet.models.Group}\pysiglinewithargsret{\strong{class }\code{ubet.models.}\bfcode{Group}}{\emph{*args}, \emph{**kwargs}}{}
Um grupo e uma coleção na qual ocorrem as apostas.
\index{active\_groups() (ubet.models.Group static method)}

\begin{fulllineitems}
\phantomsection\label{modules/models:ubet.models.Group.active_groups}\pysiglinewithargsret{\strong{static }\bfcode{active\_groups}}{\emph{user}, \emph{waiting=False}}{}
Mostra os grupos ativos para um usuario. 
Por padrao, mostra apenas os que ele nao esta incluso. 
Se waiting for True, retorna apenas os que ele esta incluso.

\end{fulllineitems}

\index{add\_user() (ubet.models.Group method)}

\begin{fulllineitems}
\phantomsection\label{modules/models:ubet.models.Group.add_user}\pysiglinewithargsret{\bfcode{add\_user}}{\emph{user}, \emph{position}}{}
adiciona o usuario no grupo numa determinada posicao. Nao confundir com bet()

\end{fulllineitems}

\index{available\_positions() (ubet.models.Group method)}

\begin{fulllineitems}
\phantomsection\label{modules/models:ubet.models.Group.available_positions}\pysiglinewithargsret{\bfcode{available\_positions}}{}{}
retorna uma lista com os indices (comecando de 1) de quais posicoes estao abertas
para aposta no grupo.

\end{fulllineitems}

\index{bet\_value (ubet.models.Group attribute)}

\begin{fulllineitems}
\phantomsection\label{modules/models:ubet.models.Group.bet_value}\pysigline{\bfcode{bet\_value}\strong{ = None}}
valor que se paga para entrar no grupo

\end{fulllineitems}

\index{creator (ubet.models.Group attribute)}

\begin{fulllineitems}
\phantomsection\label{modules/models:ubet.models.Group.creator}\pysigline{\bfcode{creator}}
por conta dos testes, um grupo pode ficar sem criador.
Como um criador nao possui papel central na aposta, essa caracteristica
foi mantida.

\end{fulllineitems}

\index{cur\_size() (ubet.models.Group method)}

\begin{fulllineitems}
\phantomsection\label{modules/models:ubet.models.Group.cur_size}\pysiglinewithargsret{\bfcode{cur\_size}}{}{}
numero de usuarios presentes no grupo

\end{fulllineitems}

\index{get\_prize() (ubet.models.Group method)}

\begin{fulllineitems}
\phantomsection\label{modules/models:ubet.models.Group.get_prize}\pysiglinewithargsret{\bfcode{get\_prize}}{}{}
retorna o valor do prêmio.

\end{fulllineitems}

\index{groups\_by\_user() (ubet.models.Group static method)}

\begin{fulllineitems}
\phantomsection\label{modules/models:ubet.models.Group.groups_by_user}\pysiglinewithargsret{\strong{static }\bfcode{groups\_by\_user}}{\emph{user}}{}
retorna uma lista com os grupos aos quais o usuario faz parte

\end{fulllineitems}

\index{max\_size (ubet.models.Group attribute)}

\begin{fulllineitems}
\phantomsection\label{modules/models:ubet.models.Group.max_size}\pysigline{\bfcode{max\_size}\strong{ = None}}
quantidade maxima de usuarios no grupo

\end{fulllineitems}

\index{nicks\_by\_group() (ubet.models.Group method)}

\begin{fulllineitems}
\phantomsection\label{modules/models:ubet.models.Group.nicks_by_group}\pysiglinewithargsret{\bfcode{nicks\_by\_group}}{}{}
Retorna uma lista com os usernames dos usuarios participantes do grupo

\end{fulllineitems}

\index{possible\_bet() (ubet.models.Group method)}

\begin{fulllineitems}
\phantomsection\label{modules/models:ubet.models.Group.possible_bet}\pysiglinewithargsret{\bfcode{possible\_bet}}{\emph{user}}{}
retorna uma tupla: possible,reason
Se possible for True, reason e uma string vazia.
Caso contrario, reason contem uma string com justificativa de falha

\end{fulllineitems}

\index{sim\_list() (ubet.models.Group method)}

\begin{fulllineitems}
\phantomsection\label{modules/models:ubet.models.Group.sim_list}\pysiglinewithargsret{\bfcode{sim\_list}}{}{}
retorna uma lista simulada do grupo.
Uma lista simulada e uma lista de tamanho igual ao numero maximo de usuarios do grupo.
Se x e y sao usuarios nas posicoes 2  e 4 de um grupo de tamanho 5, entao o metodo retornara
a seguinte lista:
{[}None,x,None,y,None{]}
\begin{quote}

.
\end{quote}

\end{fulllineitems}

\index{status\_list (ubet.models.Group attribute)}

\begin{fulllineitems}
\phantomsection\label{modules/models:ubet.models.Group.status_list}\pysigline{\bfcode{status\_list}\strong{ = ((`END', `FINISHED'), (`ABORT', `CANCELED'), (`WAIT', `WAITING'))}}
Grupo finalizado: O grupo tem a maxima quantidade de usuarios e o sorteio foi realizado
Grupo abortado: O grupo nao atingiu a maxima quantidade de usuarios antes do prazo de tempo e foi
cancelado
Grupo em espera: O grupo esta ativo e esperando usuarios

\end{fulllineitems}

\index{time\_left() (ubet.models.Group method)}

\begin{fulllineitems}
\phantomsection\label{modules/models:ubet.models.Group.time_left}\pysiglinewithargsret{\bfcode{time\_left}}{}{}
retorna o tempo restante do grupo em minutos.

\end{fulllineitems}

\index{total\_active\_groups() (ubet.models.Group static method)}

\begin{fulllineitems}
\phantomsection\label{modules/models:ubet.models.Group.total_active_groups}\pysiglinewithargsret{\strong{static }\bfcode{total\_active\_groups}}{}{}
Mostra todos os grupos que estao ativos no momento

\end{fulllineitems}

\index{update() (ubet.models.Group method)}

\begin{fulllineitems}
\phantomsection\label{modules/models:ubet.models.Group.update}\pysiglinewithargsret{\bfcode{update}}{}{}
O metodo update e chamado no grupo sempre que se deseja atualiza-lo 
para verificar se o grupo atingiu a idade maxima ou numero maximo de membros.
Se o grupo ficar cheio, um usuario dentro dele e sorteado para ser o vencedor, e o
premio lhe e dado. 
Se o grupo ficar velho, os creditos sao extornados aos usuarios

\end{fulllineitems}

\index{users\_by\_group() (ubet.models.Group method)}

\begin{fulllineitems}
\phantomsection\label{modules/models:ubet.models.Group.users_by_group}\pysiglinewithargsret{\bfcode{users\_by\_group}}{}{}
retorna uma tupla com duas listas: a primeira componente possui uma lista 
com os usuarios daquele grupo. A segunda componente e uma lista com as posicoes ocupadas 
pelos usuarios (comecando de 1). O isemo usuario na lista user\_list esta na posicao
determinada pelo isemo elemento em position\_list

\end{fulllineitems}


\end{fulllineitems}

\index{Group\_link (class in ubet.models)}

\begin{fulllineitems}
\phantomsection\label{modules/models:ubet.models.Group_link}\pysiglinewithargsret{\strong{class }\code{ubet.models.}\bfcode{Group\_link}}{\emph{*args}, \emph{**kwargs}}{}
um link descreve as participacoes de usuarios em um grupo. Nao é possivel ter
dois usuarios numa mesma posicao, nem um mesmo usuario em duas posicoes no mesmo grupo.
Essa classe representa o agregamento em SQL, e portanto so existe para representar 
um banco relacional

\end{fulllineitems}

\index{Notification (class in ubet.models)}

\begin{fulllineitems}
\phantomsection\label{modules/models:ubet.models.Notification}\pysiglinewithargsret{\strong{class }\code{ubet.models.}\bfcode{Notification}}{\emph{*args}, \emph{**kwargs}}{}
Uma notificacao e uma classe que representa que ha um aviso para o usuario de que
um dos grupos dos quais ele participa mudou de status. A linha correspondente na tabela
deve ser apagada quando o usuario visualizar a notificacao

\end{fulllineitems}

\index{Ubet\_user (class in ubet.models)}

\begin{fulllineitems}
\phantomsection\label{modules/models:ubet.models.Ubet_user}\pysiglinewithargsret{\strong{class }\code{ubet.models.}\bfcode{Ubet\_user}}{\emph{*args}, \emph{**kwargs}}{}
Classe que faz ligacao com classe User do django. 
user.first\_name eh na verdade o nome completo
\index{bet() (ubet.models.Ubet\_user method)}

\begin{fulllineitems}
\phantomsection\label{modules/models:ubet.models.Ubet_user.bet}\pysiglinewithargsret{\bfcode{bet}}{\emph{group}, \emph{position}}{}
levanta excecao se alguma condicao for quebrada durante o processo
tenha em mente que uma aposta pode estar autorizada no inicio da chamada mas
nao mais no fim da chamada da funcao

retorna sucess,reason, onde
sucess: true/false, se a aposta foi concluida com sucesso ou nao
reason: motivo pelo qual a aposta nao ocorreu, ou ``'' caso contrario

\end{fulllineitems}


\end{fulllineitems}


Views

Sobre o \textbf{toast}
Sobre o \textbf{logger}


\chapter{Indices and tables}
\label{index:indices-and-tables}\label{index::doc}\begin{itemize}
\item {} 
\DUrole{xref,std,std-ref}{genindex}

\item {} 
\DUrole{xref,std,std-ref}{modindex}

\item {} 
\DUrole{xref,std,std-ref}{search}

\end{itemize}


\renewcommand{\indexname}{Python Module Index}
\begin{theindex}
\def\bigletter#1{{\Large\sffamily#1}\nopagebreak\vspace{1mm}}
\bigletter{u}
\item {\texttt{ubet.models}}, \pageref{modules/models:module-ubet.models}
\end{theindex}

\renewcommand{\indexname}{Index}
\printindex
\end{document}
